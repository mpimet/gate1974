\documentclass{article}
\usepackage{siunitx}
\usepackage{array}
\usepackage{booktabs}    % for \toprule, \midrule, \bottomrule
\usepackage{float}       % for [H] placement
\usepackage{caption}     % for better caption handling
\usepackage{url}
\usepackage{graphicx}
\usepackage{float}
\usepackage{subcaption}  % For subfigures (if needed)


\setlength{\parindent}{0pt}
\setlength{\parskip}{6pt}

\begin{document}

\begin{titlepage}
    \centering
    \vspace*{2cm} % Vertical space from top

    % Title
    {\LARGE\bfseries GATE Data ASCII to NetCDF Conversion\par}
    \vspace{2cm}

    {\large Prepared by René Redler (MPI-M) on February 6, 2026\par}
    \vspace{3cm}

    \includegraphics[width=0.4\textwidth]{figures/GATE1974-Logo.png}
    \vspace{1cm}

    % Optional: Add a line or separator
    \rule{\textwidth}{1pt}
    \vspace{1cm}

    % Optional: Add a note or version
    {\small Version 3.2\par}

    \vfill % Push content to bottom

    % Footer (optional)
    {\small Max Planck Institute for Meteorology (MPI-M)\par}

\end{titlepage}



\section{The ASCII Data Files}

The following tar files were provided by Julia Windmiller (MPI-M) from NOAA:
\begin{itemize}
\item 1001-1400\_19740706-19740716.tar
\item 2.00.03.101-3.00.02.104\_19740601-19740930.tar
\item 2001-2600\_19740714-19740817.tar
\item 3.00.02.104-3.31.02.101\_19740601-19740930.tar
\item 3011-4074\_19740830-19740910.tar
\item 3.31.02.101-3.33.02.101\_19740601-19740930.tar
\item 3.33.02.101-3.36.21.102\_19740601-19740930.tar
\item 3.36.21.102-3.60.02.105\_19740601-19740930.tar
\item 3.60.02.105-3.64.02.101\_19740601-19740930.tar
\item 3.64.02.101-3.69.02.104\_19740601-19740930.tar
\item 4075-4772\_19740909-19740919.tar
\item 4.34.04.110-5.00.02.101\_19740601-19740930.tar
\item 5.00.02.101-5.36.02.101\_19740601-19740930.tar
\item 5.36.02.101-5.36.02.104\_19740601-19740930.tar
\item 5.36.02.104-5.60.02.102\_19740601-19740930.tar
\item 5.36.02.104-5.60.02.102\_19740601-19740930.tar
\end{itemize}

These tar files contain ASCII data files that were extracted into
separate directories, each named according to the corresponding tar
file. The content tar files were extracted into directories with the
following names:

\begin{itemize}
\item 1001-1400\_19740706-19740716
\item 2.00.03.101-3.00.02.104\_19740601-19740930
\item 2001-2600\_19740714-19740817
\item 3.00.02.104-3.31.02.101\_19740601-19740930
\item 3011-4074\_19740830-19740910
\item 3.31.02.101-3.33.02.101\_19740601-19740930
\item 3.33.02.101-3.36.21.102\_19740601-19740930
\item 3.36.21.102-3.60.02.105\_19740601-19740930
\item 3.60.02.105-3.64.02.101\_19740601-19740930
\item 3.64.02.101-3.69.02.104\_19740601-19740930
\item 4075-4772\_19740909-19740919
\item 4.34.04.110-5.00.02.101\_19740601-19740930
\item 5.00.02.101-5.36.02.101\_19740601-19740930
\item 5.36.02.101-5.36.02.104\_19740601-19740930
\item 5.36.02.104-5.60.02.102\_19740601-19740930
\item 5.36.02.104-5.60.02.102\_19740601-19740930
\end{itemize}

Appendix E of the GATE Report No 13, The International Data Management
Plan Part I (page 121 ff.), provides detailed specifications for the
magnetic tape and file format.

\subsection{Character encoding}

Each data section is delimited by two leading and one terminating
control file. Since the embedded metadata in the data files are
sufficient for interpretation, the control files are not
analyzed. Note that two distinct character encodings were used in file
creation, as indicated in line 3 of the second control file:
\begin{quote}
\scriptsize
\begin{verbatim}
01234567890=:> /STUVWXYZ,(-JKLMNOPQR*};+ABCDEFGHI.){<ÿ                       003
\end{verbatim}
\end{quote}
indicating the use of the EBCDIC 037 character Set Layout, IBM 3270 Data Stream
\begin{quote}
\scriptsize
\begin{verbatim}
0123456789  '= /STUVWXYZ,%-JKLMNOPQR  ;&ABCDEFGHI.  +!                       003
\end{verbatim}
\end{quote}
indicating Sanitized 3270 Terminal Log character set.

While this is not important for the data itself as they contain only
numbers, the Fortran format string is affected by the different
character sets. For the DC7\_CEV aircraft e.g. the same format string
appears in two manifestations. What appears as
\begin{quote}
\scriptsize
\begin{verbatim}
1(I1,I4,I10,I5,A10,I6,I4,I5,A5,30X,15(I6,2I6,I5,2I4,5I5,2I4,2I5,2I4,4F6.1,   008
1F5.1,9I1),40X)                                                              009
\end{verbatim}
\end{quote}
written with the \textit{EBCDIC 037} character set layout is converted to 
\begin{quote}
\scriptsize
\begin{verbatim}
1%I1,I4,I10,I5,A10,I6,I4,I5,A5,30X,15%I6,2I6,I5,2I4,5I5,2I4,2I5,2I4,4F6.1,   008
1F5.1,9I1<,40X<                                                              009
\end{verbatim}
\end{quote}
with the \textit{Sanitized 3270 Terminal Log} character set.

\subsection{Sorting of files}

Each ASCII file contains two header sections with various metadata,
which were used to pre-select and sort the files. Using the
\texttt{GATEsort.sh} script, each file is scanned for specific
platform types: \texttt{BLLN}, \texttt{BUOY}, \texttt{SHIP},
\texttt{A/C}, and \texttt{ARCR}. The platform name, indicating the ship
or aircraft from which data were collected or devices launched, is
extracted from the headers. Based on these types, dedicated
directories are created, with subdirectories for each individual
platform. The \texttt{A/C} and \texttt{ARCR} types are combined into a
single \texttt{AIRCRAFT} directory. \texttt{BLLN} files are copied to
the \texttt{RADIOSONDE} directory, \texttt{SHIP} files are copied to
\texttt{DSHIP}, and \texttt{BUOY} files are placed in a dedicated
\texttt{BUOY} directory.


Not every type-platform combination contains a homogeneous set of
files. To address this, the Fortran format string present in each data
file—used during its original writing—was extracted and used to
generate an MD5 checksum. The first six digits of this checksum were
then used as the name for a subdirectory. Files were subsequently
sorted into these directories based on their format string. It was
observed that files written with the same format string consistently
contain related data sets, meaning they share identical variable
precision, ordering, and variable sets.

However, in rare cases such as with the DC7 data (see
above) identical data layouts were written using two different
character encodings, resulting in distinct MD5 checksums and,
consequently, files being placed into separate subdirectories. While
these discrepancies were easily identifiable, for the sake of
simplicity, we did not implement additional checks or merge such
directories, as the differences were minimal and readily apparent.

\section{Metadata}

Apart from the type, platform and format discussed above, the first 24
lines of each data file keep information about the sampling interval,
start and end positions and time for the measurement contained in this
file and some first description of the data itself. This is followed
by multiples of 24 lines describing the physical fields contained in
the data file one by one together with units and scaling factors.

Common to all ASCII files, the physical units used to write the data
are \si{\milli\bar} for pressure, \si{\degreeCelsius} for all
temperature-related variables, \si{\metre\per\second} for velocity (or
wind), \si{\watt\per\square\metre} for radiation, and
\si{\gram\per\kilogram} for specific humidity, and height in
\si{\metre}. For writing to NetCDF, all units are converted to SI
units which requires temperature in \si{\kelvin}, specific humidity as
\si{\kilogram\per\kilogram}, and pressure as \si{\pascal}).

Time information is converted to seconds since the start of the
measurement, which is available from the metadata.

\section{Data section}

Since the original data were written using Fortran programs, it was
most natural to begin the data conversion process in Fortran as well,
enabling direct reuse of the format strings. For certain data sets,
however, the conversion programs were rewritten in C++. In these
cases, careful attention is required to precisely align the 24×80
character records with the appropriate memory layout—ensuring correct
parsing of integers and floating-point values prior to NetCDF
conversion. The C++ implementation produces bit-identical NetCDF
output when compared to the reference Fortran version, making it a
reliable and portable alternative. As such, it serves as a valuable
blueprint for users unfamiliar with Fortran who wish to convert
additional data sets or revisit previously processed data.

For each homogeneous data set, a dedicated program was developed to
read the data line by line and file by file. Each data record spans 24
lines, with each line containing 80 characters. Every record begins
with four integers, followed by a sequence of individual
measurements. While the low-level reading of lines follows a
consistent structure across all files, interpreting the data requires
human expertise based on metadata. Although the metadata are described
to some extent according to the GATE standard, this standard does not
ensure fully self-describing data. For example, quality control
parameters are stored in a non-unique manner, and time
information—along with its format—varies significantly between data
sets. These inconsistencies are detailed in the specific sections
below.

\subsection{Balloon/Radiosonde Data (BLLN)}

VLF Radiosonde data are available from research vessels
\texttt{BIDASSOA},
\texttt{CHARTERER},
\texttt{DALLAS},
\texttt{ENDURER},
\texttt{GILLISS},
\texttt{METEOR},
\texttt{OCEANOGRPR},
\texttt{QUADRA},
\texttt{RESEARCHER}, and
\texttt{VANGUARD}.

Although the data are provided by different ship crews and use varying
character sets resulting in different MD5 checksums they are all
recorded in the same format.

\begin{table}[H]
\centering
\caption{Radiosonde data}
\label{tab:radiosonde-names}
\begin{tabular}{lll}
\toprule
\textbf{Name in ASCII File} & \textbf{NetCDF Variable Name} & \textbf{Standard Name} \\
\midrule
TIME\_AFT\_LAUNCH  & flight\_time & \\
PRESSURE           & p        & air\_pressure\\
ALTITUDE           & altitude & geopotential\_height\\
TEMPERATURE        & ta       & temperature\\
TEMP\_ERROR        & ta\_err  & temperature\_error\\
SPECIFIC\_HUMDTY   & q        & specific\_humidity \\
HUMDTY\_ERROR      & q\_err   & specific\_humidity\_error \\
WIND\_VEL\_U\_COMP & u        & eastward\_wind \\
U\_COMP\_ERROR     & u\_err   & eastward\_wind\_error \\
WIND\_VEL\_V\_COMP & v        & northward\_wind \\
V\_COMP\_ERROR     & v\_err   & northward\_wind\_error \\
\bottomrule
\end{tabular}
\end{table}

Launch start and end positions are written as attributes into the NetCDF file.
Missing data are set to something like 999 or similar, and while converting
the measurements we apply a \texttt{\_FillValue} to those data when
writing the NetCDF files. Thus it can happen that all data for a level
are missing while the level itself is still kept in the vertical
axis. For each data record the time is given in seconds after lauch
which can easily converted into a NetCDf time axis expressed as
"seconds since'' lauch time. 

\begin{figure}[htbp]
    \centering
    \includegraphics[width=0.9\textwidth, height=6cm, keepaspectratio]{figures/radiosonde_positions.png}
    \caption[Radiosonde positions]{%
        Ship positions by ship name. Each color represents a different ship. 
        The scatter plot shows the geographic distribution of
        radiosonde launches over time.
    }
    \label{fig:radiosonde_positions}
\end{figure}

Radiosonde data from
\texttt{LA\_PERLE},
\texttt{CAPRICORNE}, and
\texttt{JEAN\_CHARCOT}
have not yet been processed as each of them contains only 2 files,
even though all of these 6 files have the identical format but
different from the above, with dew point temperature rather than
specific humidity, and unlike the upper set of data no error estimates
for the individual variables. Radiosonde data in \texttt{CCGS\_QUADRA}
even come with 7 different formats each with one or two files.

\subsection{Buoy Data}

\subsubsection*{Reserch Vessel Meteor (\texttt{METEOR})}

\begin{table}[H]
\centering
\caption{Meteor Buoy Data}
\label{tab:meteor-buoy-names}
\begin{tabular}{llp{3cm}}
\toprule
\textbf{Name in ASCII File} & \textbf{NetCDF Variable Name} & \textbf{Standard Name} or Quality check\\
\midrule
HOUR.MINUTE     & time & \\
DAY             & time & \\
WINDSPEED       & ws   & wind\_speed \\
VAL.FLAG WINDSP &      & if == 1 \\
WIND-DIRECTION  & wd   & wind\_from\_direction\\
VAL.FLAG DIRECT &      & if == 1  \\
DRY BULB TEMP.  & dbt  & dry\_bulb\_temperature \\
VAL.FLAG TEMP   &      & if == 1 \\
SPEL. HUMIDITY  & q    & specific\_humidity \\
VAL.FLAG HUMID. &      & if == 1 \\
WATER TEMP.     & sst  & sea\_surface\_temperature \\
VAL.FLAG WATER. &      & if == 1 \\
\bottomrule
\end{tabular}
\end{table}

Time is given as a float in format HH.MM with no seconds, the day is
provided as float but without decimals, representing the day of the year.

\subsubsection*{\texttt{HYDROGRAPHIC\_SHIP}}

\begin{table}[H]
\centering
\caption{Hydrographic Ship Buoy Data}
\label{tab:hydro-buoy-names}
\begin{tabular}{llp{3cm}}
\toprule
\textbf{Name in ASCII File} & \textbf{NetCDF Variable Name} & \textbf{Standard Name} or Quality check\\
\midrule
SEQUENCE NUMBER &     & \\
QUALITY CONTROL &     & \\
DATE OF OB      &     & \\
TIME OF OB      &     & \\
WIND-DIRECTION  & wd  & wind\_from\_direction \\
WINDSPEED       & ws  & wind\_speed \\
DRY BULB TEMP   & dbt & dry\_bulb\_temperature \\
S/L PRESSURE    & psl & sea\_level\_pressure \\
\bottomrule
\end{tabular}
\end{table}

The date of the observation in integer format is given as in day of
the year while time is provided as a character HH:MM representing hour
and minute. Seconds are assumed to be zero.  The quality control
parameter is provided as a character string indicating the quality for
time, wind direction, wind spee and the four data records. A digit is
set to 0, 1 or 2, depending on whether the data are good, suspect or
meaningless. Here we consider only good data. If time is not reliable
we discard the whole record, otherwise individual variables are
flagged with a \texttt{\_FillValue} if indicated by the quality
parameter.

\subsection{Ship Data}

In the converted dataset ship borne measurements are available in 
\texttt{DALLAS},
\texttt{FAY},
\texttt{FAYE},
\texttt{JAMES\_M\_GILLISS},
\texttt{METEOR},
\texttt{PLANET}, and
\texttt{RESEARCHER}


\subsubsection{Research Vessels James M. Gilliss
  (\texttt{JAMES\_M\_GILLISS}),  Dallas \texttt{DALLAS}, and NOAAS Researcher (\texttt{RESEARCHER})}

We first focus on hourly averages of surface meteorological data in
fdfbef available for all three ships.

\begin{table}[H]
\centering
\caption{Gilliss, Dallas and Researcher Ship Data}
\label{tab:gillis-names}
\begin{tabular}{lll}
\toprule
\textbf{Name in ASCII File} & \textbf{NetCDF Variable Name} & \textbf{Standard Name}\\
\midrule
DATE            &  & \\
TIME            &  time & \\
LATITUDE        &  lat  & \\
LONGITUDE       &  lon  & \\
SHIP SPEED      &  & \\
SHIP HEADING    &  & \\
INC.SOL.RAD.    &  rsds & incoming\_solar\_radiation \\
REFL.SOL.RAD.   &  rlus & reflective\_solar\_radiation\\
NET RADIATION   &  nrad & net\_radiation\\
KOLLS.PRESSURE  &  & \\
ROSE.PRESSURE   &  p    & air\_pressure \\
SEA SURF.TEMP.  &  sst  & sea\_surface\_temperature \\
DRY BULB TEMP.  &  & \\
WET BULB TEMP.  &  & \\
SPEC.HUMIDITY 1 &  q    & specific\_humidity \\
SPEC.HUMIDITY 2 &  & \\
DEW PT.TEMP. 1  &  & \\
DEW PT.TEMP. 2  &  & \\
WIND SPEED-BOOM &  & \\
WIND SPEED-MAST &  & \\
WIND DIR.-BOOM  &  & \\
WIND DIR.-MAST  &  & \\
WIND,U-COM,BOOM &  & \\
WIND,V-COM,BOOM &  & \\
WIND,U-COM,MAST &  & \\
WIND,V-COM,MAST &  & \\
\bottomrule
\end{tabular}
\end{table}

Except for dew point temperature, specific humidity and wind u- and
v-components a number of counts used to compute the hourly averages is
provided.

In the metadata section it is stated that "the latitude and longitude
in each data cycle define the postion at the nominal time assigned to
that record and are not averages.''

Date and time are encoded as integer YYMMDD and HHMMSST which are
converted to "seconds\_since'' the start of the measurements of the
repsective data set (file). As no quality parameter are provided we
discarded all records with unreasonable positions and times. Time has
to increase from record to record while latitudes and longitudes are
required to stay within 180\si{\degree}W-180\si{\degree}E and
90\si{\degree}S-90\si{\degree}N. Invalid data (values lower
than zero) are discarded.

From Dallas, files are processed with 3-minute and hourly data. The
sampling intervall has been appended to the NetCDF file name. When
concattenating files these two goups have to separated.

From James M. Gilliss and Researcher, 4 different sampling rates are
available, 3 minutes, 10 minutes, 30 minutes, and hourly.

Boundary layer meteorology data with 4 second instant, 3 minute and
hourly averages are available in subdirectories 0f8f15, 1b4578,
1b4578, 1d2616, 1da09a, 1e90eb, 2707b8, 270e2a, 2abe9e, 69f232,
97c87c, b32413, cd6756, f5ec40, fed03c, and ffed68.

\begin{figure}[htbp]
    \centering
    \includegraphics[width=0.9\textwidth, height=6cm, keepaspectratio]{figures/ship_positions.png}
    \caption[Ship positions]{%
        Ship positions by ship name. Each color represents a different ship. 
        The scatter plot shows the geographic distribution of measurements over time.
    }
    \label{fig:ship_positions}
\end{figure}

\subsubsection{Research Vessels Meteor (\texttt{METEOR}), H.J.W. FAY (\texttt{FAY}, \texttt{FAYE}) and Planet (\texttt{PLANET})}

Here we first focus on subdirectory 1a7095, available for all three vessels.

\begin{table}[H]
\centering
\caption{Meteor, Fay and Planet Ship Data}
\label{tab:meteor-names}
\begin{tabular}{lll}
\toprule
\textbf{Name in ASCII File} & \textbf{NetCDF Variable Name} & \textbf{Standard Name} \\
\midrule
DAY OF EVENT   &      & \\
TIME OF EVENT  & time & \\
LATITUDE       & lat  & \\
LONGITUDE      & lon  & \\
 PRESSURE      & p    & air\_pressure \\
TEMPERATURE    & ta   & air\_temperature \\
WET-BULB TEMP  &  & \\
WATER TEMP     & sst  & sea\_surface\_temperature \\
WIND DIRECTION &  & \\
WINDSPEED      &  & \\
TOTAL CLOUD    &  & \\
LOW CLOUD      &  & \\
MIDDLE CLOUD   &  & \\
HIGH CLOUD     &  & \\
RAIN DURATION  &  & \\
RAIN AMOUNT    &  & \\
\bottomrule
\end{tabular}
\end{table}

Date and time are encoded as float YYMMDD and HHMMSST which are
converted to "seconds\_since'' the start of the measurements of the
repsective data set (file). As no quality parameter are provided we
discarded all records with unreasonable positions and times. Time is
to increase from record to record while latitudes and longitudes are
required to stay within 180\si{\degree}W-180\si{\degree}E and
90\si{\degree}S-90\si{\degree}N. Invalid data values (values outside
of a reasonable data range (pressure larger than 0 mbar and temperatures
between 0 and 50\si{\degreeCelsius}). (Data are converted to SI units when
written to the NetCDF file.

For Meteor, other data sets like 1a7095, 25f518, a72c5a, or 7b0654,
focus on triples pressure, geopotential and irradiance; others like
6810fa on oceanographic parameters like waves and swell. 517055,
8b4723, c9263e, and df1d53 contain reflected and incoming shortwave
radiation, net global downward and terrestrial longwave radiation.

\subsubsection{Sea Surface Temperature from GATE and Commercial
  Ships (\texttt{GATE\_AND\_COMM\_SHIPS})}

The \texttt{GATE\_AND\_COMM\_SHIPS} directory contains post-processed
sea surface temperature data that were collected by commercial as well
as GATE ships. Each file contains a daily mean on a half degree mesh
covering the domain between 106\si{\degree}W-62\si{\degree}E and
22\si{\degree}S-38\si{\degree}N. In the metadata section it is written
that "the mesh was built from raw data by using the relaxation method
and three-day averaging (each day except the first and 100th day was
averaged with its preceeding and succeeding day for each final mesh
value).'' In the NetCDF files the time is set to 12:00:00, the middle
of the day, while in the ASCII files the time is set to 00:00:00.

\subsection{Aircraft Data}

For airborne measurements we first focus on the time averaged data
from 6 aircraft platforms.

NCAR Electra (\texttt{NCAR\_ELECTRA\_MEANS}),
NOAA DC-6 39 Charlie (\texttt{NOAA\_DC-6\_MEANS}),
NCAR Queen Air (\texttt{NCAR\_QUEEN\_AIR\_MEANS}),
NASA Convair 990 (\texttt{NASA\_CONVAIR\_990\_MEANS}),
Lockheed C-130B Hercules (\texttt{NOAA\_US-C130\_MEANS}),
and the high flying NCAR Sabreliner (\texttt{NCAR\_SABRE\_MEANS}).

For all files we require time
to increase from record to record while latitudes and longitudes are
required to stay within 180\si{\degree}W-180\si{\degree}E and
90\si{\degree}S-90\si{\degree}N. Otherwise records are discarded.

\subsubsection{NCAR Electra (\texttt{NCAR\_ELECTRA\_MEANS})}

One minute mean data were computed from 1 second data, quality flags
from 1 to 7 indicate how many data were used for the mean, 9 indicates
missing values.

Time is indicated as float/real HHMMSS provided as integer.

\begin{figure}[htbp]
    \centering
    \includegraphics[width=0.9\textwidth, height=6cm, keepaspectratio]{figures/electra-mean-positions.png}
    \caption[NCAR Electra flight tracks]{%
        NCAR Electra flight tracks. Colors represent different
        altitudes, black: 600-500 hPa, blue: 500-400 hPa, green: 400-300
        hPa, yellow: 300-200 hPa, and red: above 200 hPa.
    }
    \label{fig:electra-mean-positions}
\end{figure}

\begin{table}[H]
\centering
\caption{NCAR Electra 1-minute mean}
\label{tab:electra-mean-names}
\begin{tabular}{lll}
\toprule
\textbf{Name in ASCII File} & \textbf{NetCDF Variable Name} & \textbf{Standard Name}\\
\midrule
TIME IN GMT      & time     & \\
LATITUDE         & lat      & \\
LONGITUDE        & lon      & \\
STATIC  PRESSRE  & p        & air\_pressure \\
SD OF PRESSURE   & std\_p   & standard\_deviation\_of\_air\_pressure \\
RADAR ALTITUDE   & z        & altitude \\
SD OF RADAR ALT  & std\_z   & standard\_deviation\_of\_altitude \\
AMBNT TEMP BOOM  &  & \\
SD OF BOOM TEMP  &  & \\
AMBIENT TEMP     & ta       & air\_temperature \\
SD OF TEMP       & std\_ta  & standard\_deviation\_of\_air\_temperature \\
DEW POINT TEMP   & dew      & dew\_point\_temperature \\
SD OF DEW POINT  & std\_dew & standard\_deviation\_of\_dew\_point \\
U WIND COMPNT    &  & \\
SD OF U WIND     &  & \\
V WIND COMPNT    &  & \\
 SD OF V WIND    &  & \\
 W WIND COMPNT   &  & \\
 SD OF W WIND    &  & \\
 INS TRUE HEADG  &  & \\
 SD OF HEADING   &  & \\
 TRUE AIRSPEED   &  & \\
 INS PITCH ANGL  &  & \\
 INS ROLL ANGLE  &  & \\
 LONG RAD DOWN4  &  & \\
 SHORT RAD DOWN1 &  & \\
 SHORT RAD DOWN2 &  & \\
 SHORT RAD DOWN3 &  & \\
 LONG RAD UP4    &  & \\
 SHORT RAD UP1   &  & \\
 SHORT RAD UP2   &  & \\
 SHORT RAD UP3   &  & \\
 QUALITY FLAGS   &  & \\
\bottomrule
\end{tabular}
\end{table}

For writing the NetCDF file we check for respective missing value and
set those to a \texttt{\_FillValue}. Records with non-valid positions
are discarded as well records going back in time.



\subsubsection{NOAA DC-6 39 Charlie (\texttt{NOAA\_DC-6\_MEANS}) and Lockheed C-130B Hercules (\texttt{NOAA\_US-C130\_MEANS})}

\begin{figure}[htbp]
    \centering
    \includegraphics[width=0.9\textwidth, height=6cm, keepaspectratio]{figures/dc6-mean-positions.png}
    \caption[NOAA DC-6 39 Charlie flight tracks]{%
        NOAA DC-6 39 Charlie flight tracks. Colors represent different
        altitudes, black: 600-500 hPa, blue: 500-400 hPa, green: 400-300
        hPa, yellow: 300-200 hPa, and red: above 200 hPa.
    }
    \label{fig:dc6-mean-positions}
\end{figure}

\begin{figure}[htbp]
    \centering
    \includegraphics[width=0.9\textwidth, height=6cm, keepaspectratio]{figures/us-c130-mean-positions.png}
    \caption[Lockheed C-130B Hercules flight tracks]{%
        Lockheed C-130B Hercules flight tracks. Colors represent different
        altitudes, black: 600-500 hPa, blue: 500-400 hPa, green: 400-300
        hPa, yellow: 300-200 hPa, and red: above 200 hPa.
    }
    \label{fig:c130-mean-positions}
\end{figure}


One minute mean data were computed from 1 second data, quality flags
from 1 to 7 indicate how many data were used for the mean, 9 indicates
missing values.

Time is indicated as float/real HHMMSS provided as integer.

\begin{table}[H]
\centering
\caption{NOAA DC-6 39 Charlie and Lockheed C-130B Hercules 1-minute mean}
\label{tab:dc6-c130-mean-names}
\begin{tabular}{lll}
\toprule
\textbf{Name in ASCII File} & \textbf{NetCDF Variable Name} & \textbf{Standard Name}\\
\midrule
TIME IN GMT     & time     & \\
LATITUDE        & lat      & \\
LONGITUDE       & lon      & \\
STATIC  PRESSRE & p        & air\_pressure \\
SD OF PRESSURE  & std\_p   & standard\_deviation\_of\_air\_pressure \\
RADAR ALTITUDE  & z        & altitude \\
SD OF RADAR ALT & std\_z   & standard\_deviation\_of\_altitude \\
AMBIENT TEMP    &  & \\
SD OF TEMP      &  & \\
DEW POINT TEMP  & dew      & dew\_point\_temperature \\
SD OF DEW POINT & std\_dew & standard\_deviation\_of\_dew\_point\_temperature \\
U WIND COMPNT   &  & \\
SD OF U WIND    &  & \\
V WIND COMPNT   &  & \\
SD OF V WIND    &  & \\
INS TRUE HEADG  &  & \\
SD OF HEADING   &  & \\
TRUE AIRSPEED   &  & \\
INS PITCH ANGL  &  & \\
INS ROLL ANGLE  &  & \\
LONG RAD DOWNWD &  & \\
SHORT RAD DOWN  &  & \\
LONG RAD UPWD   &  & \\
SHORT RAD UPWD  &  & \\
QUALITY FLAGS   &  & \\
\bottomrule
\end{tabular}
\end{table}

For writing the NetCDF file we check for respective missing value and
set those to a \texttt{\_FillValue}. Records with non-valid positions
are discarded as well records going back in time.



\subsubsection{NCAR Queen Air (\texttt{NCAR\_QUEEN\_AIR\_MEANS})}

One minute mean data were computed from 1 second data, quality flags
from 1 to 7 indicate how many data were used for the mean, 9 indicates
missing values.

Time is indicated as float/real HHMMSS provided as integer.

\begin{figure}[htbp]
    \centering
    \includegraphics[width=0.9\textwidth, height=6cm, keepaspectratio]{figures/queen-air-mean-positions.png}
    \caption[NCAR Queen Air flight tracks]{%
        NCAR Queen Air flight tracks. Colors represent different
        altitudes, black: 600-500 hPa, blue: 500-400 hPa, green: 400-300
        hPa, yellow: 300-200 hPa, and red: above 200 hPa.
    }
    \label{fig:queen-air-mean-positions}
\end{figure}

\begin{table}[H]
\centering
\caption{NCAR Queen Air 1-minute mean}
\label{tab: queen-air-mean-names}
\begin{tabular}{lll}
\toprule
\textbf{Name in ASCII File} & \textbf{NetCDF Variable Name} & \textbf{Standard Name}\\
\midrule
TIME IN GMT     & time     & \\
LATITUDE        & lat      & \\
LONGITUDE       & lon      & \\
STATIC  PRESSRE & p        & air\_pressure \\
SD OF PRESSURE  & std\_p   & standard\_deviation\_of\_air\_pressure \\
AMBIENT TEMP    & ta       & air\_temperature \\
SD OF TEMP      & std\_ta  & standard\_deviation\_of\_air\_temperature \\
REVRS FLOW TEMP &  & \\
SD OF REVS TEMP &  & \\
DEW POINT TEMP  & dew      & dew\_point\_temperature \\
SD OF DEW POINT & std\_dew & standard\_deviation\_of\_dew\_point\_temperature\\
U WIND COMPNT   &  & \\
SD OF U WIND    &  & \\
V WIND COMPNT   &  & \\
SD OF V WIND    &  & \\
W WIND COMPNT   &  & \\
SD OF W WIND    &  & \\
INS TRUE HEADG  &  & \\
SD OF HEADING   &  & \\
TRUE AIRSPEED   &  & \\
INS PITCH ANGL  &  & \\
INS ROLL ANGLE  &  & \\
QUALITY FLAGS   &  & \\
\bottomrule
\end{tabular}
\end{table}

For writing the NetCDF file we check for respective missing value and
set those to a \texttt{\_FillValue}. Records with non-valid positions
are discarded as well records going back in time.


\subsubsection{NASA Convair 990 (\texttt{NASA\_CONVAIR\_990\_MEANS})}

One minute mean data were computed from 1 second data, quality flags
from 1 to 7 indicate how many data were used for the mean, 9 indicates
missing values.

Time is provided as float/real HHMMSS.

\begin{figure}[htbp]
    \centering
    \includegraphics[width=0.9\textwidth, height=6cm, keepaspectratio]{figures/convair-mean-positions.png}
    \caption[NASA Convair 990 flight tracks]{%
        NASA Convair 990 flight tracks. Colors represent different
        altitudes, black: 600-500 hPa, blue: 500-400 hPa, green: 400-300
        hPa, yellow: 300-200 hPa, and red: above 200 hPa.
    }
    \label{fig:dc6-mean-positions}
\end{figure}


\begin{table}[H]
\centering
\caption{NASA Convair 990 1-minute mean}
\label{tab:convair-mean-names}
\begin{tabular}{lll}
\toprule
\textbf{Name in ASCII File} & \textbf{NetCDF Variable Name} & \textbf{Standard Name}\\
\midrule
TIME IN GMT     & time    & \\
LATITUDE        & lat     & \\
LONGITUDE       & lon     & \\
STATIC  PRESSRE & p       & air\_pressure \\
SD OF PRESSURE  & std\_p  & standard\_deviation\_of\_air\_pressure \\
RADAR ALTITUDE  & z       & altitude \\
SD OF RADAR ALT & std\_z  & standard\_deviation\_of\_altitude \\
AMBIENT TEMP    & ta      & air\_temperature \\
SD OF TEMP      & std\_ta & standard\_deviation\_of\_air\_temperature \\
U WIND COMPNT   &  & \\
SD OF U WIND    &  & \\
V WIND COMPNT   &  & \\
SD OF V WIND    &  & \\
INS TRUE HEADG  &  & \\
SD OF HEADING &  & \\
TRUE AIRSPEED &  & \\
INS PITCH ANGLE &  & \\
INS ROLL ANGLE &  & \\
LONG RAD DOWN &  & \\
  SHORT RAD DOWN &  & \\
  RG8 FLTR RAD &  & \\
LONG RAD UP &  & \\
  SHORT RAD UP &  & \\
  RG8 FLTR RAD &  & \\
QUALITY FLAGS &  & \\
\bottomrule
\end{tabular}
\end{table}

For writing the NetCDF file we check for respective missing value and
set those to a \texttt{\_FillValue}. Records with non-valid positions
are discarded as well records going back in time.



\subsubsection{NCAR Sabreliner (\texttt{NCAR\_SABRE\_MEANS})}

One minute mean data were computed from 1 second data, quality flags
from 1 to 7 indicate how many data were used for the mean, 9 indicates
missing values.

Time is indicated as float/real HHMMSS provided as integer.

\begin{figure}[htbp]
    \centering
    \includegraphics[width=0.9\textwidth, height=6cm, keepaspectratio]{figures/sabre-mean-positions.png}
    \caption[NCAR Sabreliner flight tracks]{%
        NASA Convair 990 flight tracks. Colors represent different
        altitudes, black: 600-500 hPa, blue: 500-400 hPa, green: 400-300
        hPa, yellow: 300-200 hPa, and red: above 200 hPa.
    }
    \label{fig:sabre-mean-positions}
\end{figure}

\begin{table}[H]
\centering
\caption{NCAR Sabreliner 1-minute mean}
\label{tab:netcdf_names}
\begin{tabular}{lll}
\toprule
\textbf{Name in ASCII File} & \textbf{NetCDF Variable Name} & \textbf{Standard Name}\\
\midrule
TIME IN GMT     & time     & \\
LATITUDE        & lat      & \\
LONGITUDE       & lon      & \\
STATIC  PRESSRE & p        & air\_pressure \\
SD OF PRESSURE  & std\_p   & standard\_deviation\_of\_air\_pressure \\
RADAR ALTITUDE  & z        & altitude \\
SD OF RADAR ALT & std\_z   & standard\_deviation\_of\_altitude\\
AMBIENT TEMP    & ta       & air\_temperature\\
SD OF TEMP      & std\_ta  & standard\_deviation\_of\_air\_temperature\\
DEW POINT TEMP  & dew      & dew\_point\_temperature \\
SD OF DEW POINT & std\_dew & standard\_deviation\_of\_dew\_point\_temperature \\
U WIND COMPNT   &  & \\
SD OF U WIND    &  & \\
V WIND COMPNT   &  & \\
SD OF V WIND    &  & \\
W WIND COMPNT   &  & \\
SD OF W WIND    &  & \\
INS TRUE HEADG  &  & \\
SD OF HEADING   &  & \\
TRUE AIRSPEED   &  & \\
INS PITCH ANGL  &  & \\
INS ROLL ANGLE  &  & \\
SHORT RAD DOWN  &  & \\
SHORT RAD UP    &  & \\
LONG RAD DOWN   &  & \\
LONG RAD UP     &  & \\
QUALITY FLAGS   &  & \\
\bottomrule
\end{tabular}
\end{table}

For writing the NetCDF file we check for respective missing value and
set those to a \texttt{\_FillValue}. Records with non-valid positions
are discarded as well records going back in time.


\subsubsection{NOAA DC-6 39 Charlie (\texttt{39\_CHARLIE})}

Files contain 1 second data cycles, quality flags indicating 1 for
good data, 2 suspect data, and 3 meaning no control.

Time is provided as integer DDDHHMMSS. The table
\ref{tab:39_charlie_names} only list the physical variable without
quality flags.

\begin{figure}[htbp]
    \centering
    \includegraphics[width=0.9\textwidth, height=6cm, keepaspectratio]{figures/39-charlie-positions.png}
    \caption[NOAA DC-6 39 Charlie flight tracks]{%
        NOAA DC-6 39 Charlie flight tracks. Colors represent different
        altitudes, black: 600-500 hPa, blue: 500-400 hPa, green: 400-300
        hPa, yellow: 300-200 hPa, and red: above 200 hPa.
    }
    \label{fig:39-charlie-positions}
\end{figure}

\begin{table}[H]
\centering
\caption{NOAA DC-6 39 Charlie 1 second data}
\label{tab:39_charlie_names}
\begin{tabular}{lll}
\toprule
\textbf{Name in ASCII File} & \textbf{NetCDF Variable Name} & \textbf{Standard Name}\\
\midrule
TIME IN GMT           & time  & \\
LATITUDE              & lat   & \\
  LONGITUDE           & lon   & \\
  PRESSURE            & p     & air\_pressure \\
  HEADING             &  & \\
  EAST WIND VE        &  & \\
  NORTH  WIND VN      &  & \\
  TWENTY W SAMPLES    &  & \\
  TWENTY V SAMPLES    &  & \\
  TWENTY U SAMPLES    &  & \\
  TWENTY RHOV SAMPLES & rhov  & water\_vapor\_density\\
  TWENTY T SAMPLES    & ta    & air\_temperature\\
\bottomrule
\end{tabular}
\end{table}

For writing the NetCDF file we discarded records going back in time.

If any of the variables in a record is flagged as suspect or no
control the whole record is written with \texttt{\_FillValue} to the
NetCDF file. In addition all data with pressure lower than 40 hPa are
set to \texttt{\_FillValue} as well. From the 20 samples per second
for air\_temperature and water\_vapor\_density we calculated averages and
standard deviation which is written to the NetCDF files..

\subsubsection{France DC-7 (\texttt{DC-7\_CEV})}

Files include two data set, one with a 1-second samling and a second
with 1-minute averages. Strings 1S and 1M have been appended to the
NetCDF file name accordingly.

ASCII files are distributed over two md4sum directories 579bd3
d30c25.

Time is provided as an integer HH MM SS, Leading 0s for HH and MM have
been omitted which makes it a bit challenging to reconstruct the
time. Furthermore, the time signal is occasionally distored leading to
backward jumps in time. Those records have been discarded.

\begin{figure}[htbp]
    \centering
    \includegraphics[width=0.9\textwidth, height=6cm, keepaspectratio]{figures/dc7-1m-positions.png}
    \caption[France DC-7 flight tracks based on 1 minute data]{%
        France DC-7 flight tracks. Colors represent different
        altitudes, black: 600-500 hPa, blue: 500-400 hPa, green: 400-300
        hPa, yellow: 300-200 hPa, and red: above 200 hPa.
    }
    \label{fig:dc7-positions}
\end{figure}

\begin{table}[H]
\centering
\caption{France DC-7 1 second and 1 minute data}
\label{tab:dc7_names}
\begin{tabular}{lll}
\toprule
\textbf{Name in ASCII File} & \textbf{NetCDF Variable Name} & \textbf{Standard Name}\\
\midrule
TIME IN GMT     & time  & \\
LATITUDE        & lat   & \\
LONGITUDE       & lon   & \\
INS TRUE HEADG. &  & \\
GRND SPEED      &  & \\
TRUE AIR SPEED  &  & \\
INS TRACK       &  & \\
PITCH ANGLE     &  & \\
ROLL ANGLE      &  & \\
U WIND COMPNT.  &  & \\
V WIND COMPNT.  &  & \\
WIND DIRECTION  &  & \\
WIND SPEED      &  & \\
STATIC PRESSURE & p     & air\_pressure \\
ALTITUDE        & z     & altitude \\
AMBIENT TEMP.   & ta    & air\_temperature \\
AMBIENT TEMP.   & ta\_2 & air\_temperature \\
SHORT RAD DOWN  &  & \\
SHORT RAD UP    &  & \\
LONG RAD DOWN   &  & \\
LONG RAD UP     &  & \\
SURF TEMPERAT.  & tas   & surface\_temperature \\
QUALITY FLAGS   &  & \\
\bottomrule
\end{tabular}
\end{table}

\subsubsection{UK C-130B (\texttt{UKHERCULES\_XV208} )}

From UK C-130B Hercules a variety of data sets exist, here we focus on
two subsets), dab4ec (as NetCDF available in
\texttt{UKHERCULES\_XV208a}) and 96dd74 (as NetCDF available in
\texttt{UKHERCULES\_XV208b}). Two data sets are included, one with a 1
minute interval (100F) and a seconbd one with a 1 second interval (1F)

\begin{figure}[htbp]
    \centering
    \begin{minipage}[t]{0.48\textwidth}
        \centering
        \includegraphics[width=\textwidth, height=6cm, keepaspectratio]{figures/uk-hercules-a-100-positions.png}
        \caption[UK C-130B flight tracks. Part 1]{%
            UK C-130B flight tracks based on 1 minute data..
        }
        \label{fig: uk-hercules-a}
    \end{minipage}
    \hfill
    \begin{minipage}[t]{0.48\textwidth}
        \centering
        \includegraphics[width=\textwidth, height=6cm, keepaspectratio]{figures/uk-hercules-b-100-positions.png}
        \caption[UK C-130B flight tracks. Part 2]{%
            As Fig. \ref{fig: uk-hercules-a} but for the second data set.
        }
        \label{fig: uk-hercules-b}
    \end{minipage}
    %\captionsetup{labelformat=empty} % removes "Figure 1a", "Figure 1b"
    %\caption{UK C-130B flight tracks based on 1 minute data].}
    %\label{fig: uk-hercules}
  \end{figure}

\begin{table}[H]
\centering
\caption{UK C-130B Hercules  1 second and 1 minute data}
\label{tab:dc7_names}
\begin{tabular}{lll}
\toprule
\textbf{Name in ASCII File} & \textbf{NetCDF Variable Name} & \textbf{Standard Name}\\
\midrule
  RECORDED GMT    & time & \\
  TRUE HEADING
  AIRSPEED
  LATITUDE        & lat  & \\
  LONGITUDE       & lon  & \\
  STATIC PRESSURE & p    & air\_pressure \\
  PRESSURE HEIGHT &  & \\
  RADAR HEIGHT    & z    & altitude \\
  D DRIFT DOPP    &  & \\
  ANGLE ATTACK    &  & \\
  ANGLE SIDESLIP  &  & \\
  PITCH ANGLE     &  & \\
  ROLL ANGLE      &  & \\
  GROUND SPEED E. &  & \\
  GROUND SPEED N. &  & \\
  A/C VERT VEL    &  & \\
  WIND TO EAST    &  & \\
  WIND TO NORTH   &  & \\
  WIND COMP VERT  &  & \\
  AIR TEMPERATURE & ta    & air\_temperature \\
  SPECIFIC HUMDTY & q     & specific\_humidity \\
  AIR TEMPERATURE & ta\_2 & air\_temperature \\
  DEW/FROST POINT & dew   & dew\_point\_temperature \\
  RADIATIVE TEMP  &  & \\
  QUALITY FLAGS   &  & \\
\bottomrule
\end{tabular}
\end{table}


\subsection{Dropsonde Data}

We found two dataset containing dropsonde measurements, one from C130
data from Lockheed C-130B Hercules
L-282\footnote{\url{https://www.airhistory.net/photo/20084/N6541C}} (\texttt{C130\_DROPSONDE})
and the other from C135 data from Air Force Boeing
OC-135B\footnote{\url{https://www.airhistory.net/photo/763834/61-2674/12674}}(\texttt{C135\_DROPSONDE})

Time is provided as integer HHMMSS.

\begin{figure}[htbp]
    \centering
    \includegraphics[width=0.9\textwidth, height=6cm, keepaspectratio]{figures/dc7-1m-positions.png}
    \caption[Dropsonde launch positions]{%
        Dropsonde launch positions.
    }
    \label{fig:dropsonde-positions}
\end{figure}

\begin{table}[H]
\centering
\caption{C-130B Dropsonde Data}
\label{tab:c130-dropsonde-names}
\begin{tabular}{lll}
\toprule
\textbf{Name in ASCII File} & \textbf{NetCDF Variable Name} & \textbf{Standard Name}\\
\midrule
 TIME        & time     & \\
 PRESSURE    & p        & air\_pressure\\
 HEIGHT      & altitude & geopotential\_height\\
 DEW POINT   & dew      & dew\_point\_temperature \\
 TEMPERATURE & ta       & temperature\\
 RLT.HUM.    & rh       & relative\_humidity \\
 SPC.HUM.    & q        & specific\_humidity \\
 DIRECTION   & wdir     & wind \_from\_direction \\
 SPEED       & wspd     & wind\_speed \\
 U-COMP      & u        & eastward\_wind \\
 V-COMP      & v        & northward\_wind \\
 LATITUDE    & lat      & \\
 LONGITUDE   & lon      & \\
\bottomrule
\end{tabular}
\end{table}

Missing values set in the ASCII files are evaluated and corresponding
data set to \texttt{\_FillValue}. Data are furthermore checked and all
individual data carrying values outside of a plausible data range are
set to \texttt{\_FillValue} as well.

\section{A Note on the c++ implementation}

In Fortran, reading the 24×80 character records is straightforward when
using the provided format string in conjunction with a properly
defined derived type.

Taking the example for converting the Meteor buoy data (see
\texttt{GATEbuoy\_meteor.f90} and \texttt{GATEbuoy\_meteor.cpp}) the
respective Fortran code sections look like

\begin{verbatim}
  type :: GATE_buoy_type
     ! do not change the sequence within this type
     real :: TIME
     real :: DAY
     real :: WIND_SPEED
     real :: VAL_WIND_SPEED
     real :: WIND_DIRECTION
     real :: VAL_WIND_DIRECTION
     real :: DRY_BULB_TEMP
     real :: VAL_DRY_BULB_TEMP
     real :: SPEC_HUMIDITY
     real :: VAL_SPEC_HUMIDITY
     real :: WATER_TEMPERATURE
     real :: VAL_WATER_TEMPERATURE
  end type GATE_buoy_type

  integer :: type_id          ! file header record type indicator
  integer :: records_in_line  ! number of of records contained in one full data line
  integer :: records_handled  ! number of records allready stored away
  integer :: line_number      ! of one line containing full format string

  integer, parameter :: no_of_records_in_line = 25

  type (GATE_buoy_type) :: buoydata(no_of_records_in_line)

  character(len=24*80) :: line

  110 format(I1,I4,I10,I5,25(F10.2,F5.0,F7.1,F4.0,F7.0,F4.0,F9.2,F4.0, &
             F9.2,F4.0,F9.2,F4.0))

  read(line, 110, iostat=ierror ) &
        type_id, records_in_line, records_handled, line_number, buoydata
\end{verbatim}

The structure of the data where a specific
format is repeated multiple times can be efficiently captured by
declaring arrays with appropriate dimensions.
For example, the variable \texttt{radiosondedata} is declared as an
array of 25 elements to reflect that the format
\begin{verbatim}
(F10.2,F5.0,F7.1,F4.0,F7.0,F4.0,F9.2,F4.0,F9.2,F4.0,F9.2,F4.0)
\end{verbatim}
is repeated 25 times. This principle, using array dimensions to match
repeated format patterns, is general and applies to all GATE ASCII
files.

The c++ equivalent of a Fortran derived type is a c struct. It is not
possible to implictly or even explicitly loop over the members of a
struct. Thus the single line in Fortran has to be expanded manually
into a squence where we read the 24x80 character string variable by
variable and updating the pointer to the string after each read.

\begin{verbatim}
  constexpr int BUOY_DATA_IN_LINE = 25;

  std::vector<GATE_buoy_type> buoydata(BUOY_DATA_IN_LINE);

  std::sscanf(record_lines.c_str(), "%1d %4d %10d %5d",
              &type_id,
              &records_in_line,
              &records_handled,
              &line_number);

  const char *ptr = record_lines.c_str() + 20;

  const char* const format = "%10f %5f %7f %4f %7f %4f %9f %4f %9f %4f %9f %4f";

  for (int i = 0; i < BUOY_DATA_IN_LINE; i++) {
    std::sscanf(ptr,
                format,
                &buoydata[i].time,
                &buoydata[i].day,
                &buoydata[i].wind_speed,
                &buoydata[i].val_wind_speed,
                &buoydata[i].wind_direction,
                &buoydata[i].val_wind_direction,
                &buoydata[i].dry_bulb_temp,
                &buoydata[i].val_dry_bulb_temp,
                &buoydata[i].spec_humidity,
                &buoydata[i].val_spec_humidity,
                &buoydata[i].water_temperature,
                &buoydata[i].val_water_temperature);

    ptr += RECORD_WIDTH; //10+5+7+4+7+4+9+4+9+4+9+4
  }
\end{verbatim}

It may become necessary in c++ to write the loop even more explicitly
when the Format string is not as homogenous as in the Meteor buoy case
or if numbers are not seperated by blanks anymore as in the latter
case the sscanf will fail. The alternative for loop would look like

\begin{verbatim}
  for (int i = 0; i < BUOY_DATA_IN_LINE; i++)
    {
      std::sscanf(ptr, "%10f", &buoydata[i].time);
      ptr += 10;

      std::sscanf(ptr, "%5f", buoydata[i].day);
      ptr += 5;

      std::sscanf(ptr, "%7f", buoydata[i].wind_speed);
      ptr += 7;

      ...
    }
\end{verbatim}

See \texttt{GATEbuoy\_hydro.cpp} for a more advanced example.

\end{document}
